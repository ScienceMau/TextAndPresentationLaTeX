\documentclass[12pt,aspectratio=169]{beamer}
\usetheme{Berlin}
\usepackage[utf8]{inputenc}
\usepackage[portuguese]{babel}
\usepackage[T1]{fontenc}
\usepackage{amsmath}
\usepackage{amsfonts}
\usepackage{amssymb}
\usepackage{graphicx}

\setbeamertemplate{footline}[frame number] 

\author[Dr. M.A.Ribeiro ]{Dr.Mauricio A. Ribeiro}
\title[Ribeiro, M. A.]{Breve introdução Sistemas Dinâmicos e suas principais métricas}
%\setbeamercovered{transparent} 
%\setbeamertemplate{navigation symbols}{UTFPR-PG} 
\logo{\includegraphics[scale=0.05]{logo}} 
\institute[UTFPR-PG]{Universidade Tecnológica Federal do Paraná - Campus Ponta Grossa} 
%\date{} 
%\subject{} 

%%%%%%%%%%%%%%%%%%%%%%%%%%%%%%%%%%%%%%%%%%%%%%%%%%%%%%%%%%%%%%%%%%%%%%%%

%%%%%%%%%%%%%%%%%%%%%%%%%% DIVISÃO DOS TÓPICOS %%%%%%%%%%%%%%%%%%%%%%%%%
% EGMON: 
% Introdução, E a Humanidade inventa o Capitalismo, Acumulando Capital e Revolucionando a Indústria, Competição Capitalista e Barbárie Humana, Uma alternativa ao Capitalismo
%
% WASHINGTON:
% Concorrência e Monopólio, A Crise: Superprodução de Mercadoria e Imperialismo, Mas, o que é realmente Socialismo?, Tentaram, mas não conseguiram
%%%%%%%%%%%%%%%%%%%%%%%%%%%%%%%%%%%%%%%%%%%%%%%%%%%%%%%%%%%%%%%%%%%%%%%%%


\begin{document}

\begin{frame}
\titlepage
\end{frame}

\begin{frame}
\tableofcontents
\end{frame}


\section{Breve introdução Sistemas Dinâmicos}
\begin{frame}{Breve introdução Sistemas Dinâmicos}
\begin{itemize}
    \item Sistemas dinâmicos são sistemas fora do equilíbrio, caracterizados por estados que mudam com o tempo. São usados para modelar e fazer previsões de sistemas físicos, biológicos, financeiros, etc.
    
    \item Para definir um sistema dinâmico precisamos de três ingredientes:
    \begin{itemize}
        \item Espaço de estados
        \item Equações de movimento
        \item Medida de distância 
    \end{itemize}
\end{itemize}
\end{frame}


\begin{frame}
\begin{columns}
\column{0.5\textwidth}
\begin{block}{Discreto:}
$x_{n+1}=f(x_n)$
\end{block}
\begin{block}{ Contínuos:}
$\dot{x}=f(x)$
\end{block}
\begin{block}{ Campos:}
$\frac{\partial\phi(x,t)}{\partial t}=D[\phi(x,t)]+\psi(x,t)$
\end{block}
 \column{0.5\textwidth}
\begin{block}{  Autômato Celular:}
$a^{i}_{n+1}= F[\{a^i_n\}_{j\in U_i}]$
\end{block}
\begin{block}{ Redes Complexas: }
$a^{i}_{n+1} = F[\{a^{j}_{n}\}_{j \in V_i}]$
\end{block}
\end{columns}
\end{frame}



\begin{frame}
\begin{columns}
\column{0.5\textwidth}
\begin{block}{}
Sistemas Lineares e Não Lineares
\end{block}
\begin{block}{}
Conservativos e Não Conservativos
\end{block}
\begin{block}{}
Determinísticos e Probabilísticos
\end{block}
 \column{0.5\textwidth}
\begin{block}{}
Autômantos e não autômatos
\end{block}
\begin{block}{}
Com retardo e não retardo
\end{block}
\end{columns}
\end{frame}





\section{Oscilador de Duffing}
\begin{frame}{Explorar Oscilador de Duffing }
\begin{columns}
\column{0.5\textwidth}
\begin{block}{Oscilador de Duffing}
\begin{eqnarray}
\ddot{x}+\delta\dot{x}+\alpha x+\beta x^3 = f_0cos(\omega t)
\end{eqnarray}
\begin{itemize}
    \item $\alpha =0.02$, $\beta=5.0$, $\delta=0.3$, $\omega=1.0$ e $f_0 \in [0,5.0]$
    \item As condições iniciais foram $(0,0,0)$, tempo de integração $10^6[s]$ e tempo transiente de $40\%$ do tempo total.
\end{itemize}
\end{block}
\column{0.5\textwidth}
\begin{block}{Análise}
\begin{itemize}
    \item Analisar o máximo das séries temporais (Diagrama de Bifurcação) 
    \item Comportamento de duas condições iniciais próximas (Expoente de Lyapunov)
    \item Comportamento quando a interseção de planos ($h_p=\frac{2\pi n}{\omega}$) 
\end{itemize}
\end{block}
\end{columns}
\end{frame}

\begin{frame}{Diagrama de Bifurcação}
\begin{figure}[h]
\centering
\includegraphics[width=0.75\textwidth]{FIGURAS/bifur2.eps} 
\caption{Diagrama de Bifurcação}
\label{fig:02}
\end{figure}
\end{frame}

\subsection{Seção de Poincaré}
\begin{frame}{Seção de Poincaré}
\begin{itemize}
    \item Técnica permite transformar um sistema dinâmico contínuo no tempo em um sistema discreto.
    \item Dado um sistema de dimensão $n$ com órbitas periódicas em seu espaço de fase, a seção de Poincaré diminui a dimensão em $n-1$
    \item Sendo determinada realizando cortes perpendiculares as trajetórias ao espaço de fase e tomando somente os pontos que interceptam essa seção.
\end{itemize}
\end{frame}


\begin{frame}{Seção de Poincaré}
\begin{figure}[h]
\centering
\includegraphics[width=0.60\textwidth]{FIGURAS/esquema2.png} 
\caption{Ideia principal do Mapa de Poincaré}
\label{fig:03}
\end{figure}
\end{frame}


\begin{frame}{Seção de Poincaré}
\begin{figure}[h]
\centering
\includegraphics[width=0.6\textwidth]{FIGURAS/esquema1.png} 
\caption{Planos para interseção da trajetória.}
\label{fig:03}
\end{figure}
\end{frame}

\begin{frame}{Oscilador de Duffing }
    \begin{block}{}
        \begin{eqnarray}
            \ddot{x}+\delta\dot{x}+\alpha x+\beta x^3 = f_0cos(\omega t)
    \end{eqnarray}
    \begin{itemize}
        \item $\alpha =0.02$, $\beta=5.0$, $\delta=0.3$, $\omega=1.0$ e $f_0 \in [0,5.0]$
        \item As condições iniciais foram $(0,0,0)$, tempo de integração $10^6[s]$ e tempo transiente de $40\%$ do tempo total.
    \end{itemize}
    \end{block}
\end{frame}


\begin{frame}{Seção de Poincaré}
\begin{columns}
    \column{0.6\textwidth}
        \begin{figure}[h]
            \centering
            \includegraphics[width=\textwidth]{FIGURAS/poincare.eps} 
            \caption{Mapas de Fase e Seção de Poincaré}
        \end{figure}
    \column{0.4\textwidth}
        \begin{figure}[h]
            \centering
            \includegraphics[width=\textwidth]{FIGURAS/bifur2.eps} 
        \end{figure}
\end{columns}
\end{frame}

\subsection{Expoente de Lyapunov}
\begin{frame}
\begin{columns}
\column{0.4\textwidth}
\begin{figure}[h]
\centering
\includegraphics[width=\textwidth]{FIGURAS/lyapunov.png} 
\caption{Distância entre trajetórias}
\label{fig:03}
\end{figure}
\column{0.6\textwidth}
\begin{eqnarray}
\lambda = \frac{1}{t_n-t_0}\sum^{n}_{i=1}ln\left[ \frac{\delta(t_i)}{\delta_0} \right]
\end{eqnarray}
\begin{block}{Duas condições iniciais muito próximas}
\begin{itemize}
    \item Próximos convergem para um atrator periódico ($\lambda \leq 0$) (INDICATIVO DE SER PERIÓDICO)
    \item Se a distância se afastam exponencialmente ($\lambda > 0$) o sistema é sensível a uma pequena variação de condição inicial (INDICATIVO FORTE DE CAOS).
\end{itemize}
\end{block}
\end{columns}
\end{frame}


\begin{frame}
\begin{figure}[h]
\centering
\includegraphics[width=0.78\textwidth]{FIGURAS/Lyapunov2.eps} 
\caption{Diagrama de Bifurcação e Expoentes de Lyapunov}
\label{fig:03}
\end{figure}
\end{frame}

\begin{frame}{}
\begin{figure}[h]
\centering
\includegraphics[width=0.5\textwidth]{FIGURAS/lyap1.eps} 
\caption{Espaço de parâmetros do Expoente Máximo de Lyapunov $f_0 \in [0,5.0]$ e $\omega_0 \in [0,2.0]$}
\label{fig:04}
\end{figure}
\end{frame}

\begin{frame}
\begin{figure}[h]
\centering
\includegraphics[width=0.5\textwidth]{FIGURAS/lyap2.eps} 
\caption{Espaço de parâmetros do Expoente Máximo de Lyapunov  $f_0 \in [0,5.0]$ e $\omega_0 \in [0,2.0]$}
\label{fig:05}
\end{figure}
\end{frame}



\subsection{Teste 0-1}
\begin{frame}
\begin{itemize}
    \item O teste 01 é aplicado em uma série temporal;
    \item Estima o parâmetro $K_c$
    \begin{equation} \label{eq:08t01} 
p\left(n,\bar{c}\right)=\sum _{j=0}^{n}x\left(j\right)\cos \left(j\bar{c}\right)  
\end{equation} 
\begin{equation} \label{eq:09t01} 
q\left(n,\bar{c}\right)=\sum _{j=0}^{n}x\left(j\right)\sin \left(j\bar{c}\right)  
\end{equation} 
\noindent onde $\hat{c}\in \left(0,\pi \right)$ são constantes. E $p\left(n,\hat{c}\right)$ e $q\left(n,\hat{c}\right)$ é descrito por: 

\begin{equation} \label{eq:10t01} 
M(n,c)={\mathop{\lim }\limits_{n\to \infty }} \frac{1}{N} \sum _{j=1}^{N}\left[\left(p(j+n,\bar{c})-p(j,\bar{c})\right)^{2} +\left(q(j+n,\bar{c})-q(j,\bar{c})\right)^{2} \right]  
\end{equation} 
\end{itemize}
\end{frame}

\begin{frame}
\noindent onde $n=1,2,...,N$ e definimos o \textit{$K_{c}$} em um tempo longo:
\begin{equation} \label{eq:11t01} 
K_{c} =\frac{cov\left(Y,M(\bar{c})\right)}{\sqrt{var(Y)var(M(\bar{c}))} }  
\end{equation} 
\noindent logo $M(\bar{c})=\left[M(1,\bar{c}),M(2,\bar{c}),\ldots ,M(n_{\max } ,\bar{c})\right]$ and  $Y=\left[1,2,\dots ,n_{max}\right]$.
Dado dos valores $x$ e $y$, a covariancia ${cov \left(x,y\right)\ }$ e variancia ${var \left(x\right)\ }$, of $n_{max}$ elementos, são definidos:
\begin{equation} \label{eq:012t01} 
\begin{array}{l} {cov\left(x,y\right)=\frac{1}{n_{\max } } \sum _{n=1}^{n_{\max } }\left(x\left(n\right)-\bar{x}\right)\left(y\left(n\right)-\bar{y}\right) } \\ {var\left(x\right)=cov\left(x,x\right)} \end{array} 
\end{equation}
\end{frame}

\begin{frame}{}
\begin{itemize}
    \item Os valores dos parâmetro $K_c$  é obtido por  Eq.\ref{eq:11t01}.
    \begin{itemize}
        \item Se $K_{c} = 0$  o sistema é periódico.
        \item Se $K_{c} = 1$ o sistema é caótico.
    \end{itemize}
\end{itemize}
\end{frame}
\begin{frame}{}

\begin{figure}[h]
\centering
\includegraphics[width=0.85\textwidth]{FIGURAS/Zero1.eps} 
\caption{Diagrama de Bifurcação e Teste 01}
\label{fig:03}
\end{figure}
\end{frame}



\section{Outro exemplo}
\subsection{Microscopia de força atômica}
\begin{frame}[fragile]
\begin{figure}[h]
    \centering
    \includegraphics[scale=0.3]{FIGURAS/EsquemaAFM.png}
    \caption{Representação do esquema da Microscopia de Força Atômica.\href{https://www.youtube.com/watch?v=8gCf1sEn0UU}{ Click aqui para o exemplo 1}. \href{https://www.youtube.com/watch?v=s6KqJS1GZNE}{Clique aqui para o exemplo 2}}
    \label{fig:esquemaAFM}
\end{figure}
\end{frame}

\begin{frame}
\begin{block}{Equações diferenciais}
\begin{eqnarray}
\dot{x}_1&=&x_2 \\ \nonumber
\dot{x}_2&=&-x_1-d_1 x_2+B_1+ \frac{C_1}{(1-x_{1}-\zeta\sin(\Omega \tau))^8}+\frac{C2}{(1-x_{1}-\zeta\sin(\Omega \tau))^2} \\ \nonumber
&-&\frac{p}{(1-x_{1}-\zeta\sin(\Omega \tau))^8}\dot{x}_3+\zeta\Omega^2E_1\sin(\Omega \tau) \\ \nonumber
\dot{x}_3&=&x_2 \\ \nonumber
\end{eqnarray}
\end{block}
\end{frame}

\begin{frame}
    \begin{table}[h]
\caption{Propriedades do modelo}
\label{tab:2}       % Give a unique label
% For LaTeX tables use
\begin{tabular}{ll}
\hline\noalign{\smallskip}
Parameter & Values \\ \hline 
$d_1$ & 0.01 \\  
$B_1$ & -0.148967 \\  
$C_1$ & -4.59118x10${}^{-5}$ \\  
$C_2$ & 0.149013 \\  
${\zeta }$ & 0.99 \\  
$\Omega $ & 1 \\  
$E_1$ & 1.57367 \\ \hline 
\end{tabular}
\end{table}
\par As condições iniciais foram $(0,0,0)$, tempo de integração $10^6[s]$ e tempo transiente de $40\%$ do tempo total.
\end{frame}

\begin{frame}
\begin{figure}[h]
\centering
\includegraphics[scale=0.3]{FIGURAS/novobifp.eps}
\caption{Diagramas de Bifurcação para $q=(1,~1,~1)$ e $p\in[0.001,0.05]$. (a) $x_1$ e (b)  $x_2$}
\end{figure}
\end{frame}

\begin{frame}{}
    \begin{figure}[!ht]
\centering
\resizebox{\columnwidth}{!}{\includegraphics{FIGURAS/t01AeB.eps}}
\caption{Teste 0-1.(a) é $K_c$ do deslocamento $x_1$ e (b) é $K_c$ da $\dot{x}$. }
\label{fig:5}       % Give a unique label
\end{figure}
\end{frame}

\begin{frame}
\begin{columns}
\column{0.45\textwidth}
\begin{figure}[h]
\centering
\includegraphics[scale=0.35]{FIGURAS/lyapunovp.eps}
\caption{Expoente de Lyapunov}
\end{figure}
\column{0.45\textwidth}
\begin{figure}[h]
\centering
\includegraphics[scale=0.3]{FIGURAS/lyapunovZOOM.eps}
\caption{Expoente de Lyapunov considerando $\zeta \times p $}
\end{figure}
\end{columns}
\end{frame}

\begin{frame}{}
\begin{enumerate}
    \item Ribeiro, M. A., Tusset, A. M., Lenz, W. B., Kirrou, I., Balthazar, J. M. (2021). Numerical analysis of fractional dynamical behavior of Atomic Force Microscopy. The European Physical Journal Special Topics, 230(18), 3655-3661.
    \item Ribeiro, M. A., Balthazar, J. M., Lenz, W. B., Rocha, R. T., Tusset, A. M. (2020). Numerical exploratory analysis of dynamics and control of an atomic force microscopy in tapping mode with fractional order. Shock and Vibration, 2020.
    \item Balthazar, J. M., Tusset, A. M., Ribeiro, M. A., et. al, (2021). Sistemas dinâmicos e mecatrônicos-Volume 1: Teoria e aplicação de controle. Editora Blucher.
\end{enumerate}
\end{frame}

\begin{frame}{}
\begin{block}{Linguagens de programação utilizadas}
    \begin{figure}[h]
        \centering
        \includegraphics[scale=0.13]{logos/Julia.png}
        \includegraphics[scale=0.3]{logos/matlab.png}
        \includegraphics[scale=0.23]{logos/Python.png}
    \end{figure}
\end{block}

\begin{block}{Contatos:}
\begin{description}
\item[Instagram:] @science.mau
\item[Email:] mau.ap.ribeiro@gmail.com
\item[Git Hub:] https://github.com/ScienceMau

\end{description}
\end{block}
\end{frame}



\end{document}
